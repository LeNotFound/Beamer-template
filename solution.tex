\documentclass[aspectratio=169]{beamer} % 设置纵横比 16:9
% \documentclass[table]{beamer}
\usepackage[UTF8, noindent]{ctexcap}
\usepackage{graphicx, graphics}
\usepackage{float, array, color, ctex}
\usepackage{amsmath, amssymb}
\usepackage{multicol, multirow, makecell, tabu, dcolumn}
\usepackage{fancyhdr, lastpage}
\usepackage{listings, xcolor}
\usepackage{xeCJKfntef}
\usepackage{fontspec, xunicode, xltxtra}
\usepackage{setspace}
\usepackage{geometry}
\usepackage{listings}
\usepackage{hyperref}
\usetheme{Szeged}   % 使用主题
\usecolortheme{default} % 使用配色方案
\usefonttheme[onlymath]{serif}  % 使用字体主题
\setsansfont{Consolas}  % 设置正文字体
\setbeamertemplate{background}{\includegraphics[height=\paperheight]{bg}} % 设置背景图片

\title{标题}    % 标题
\author{LeNotFonud} % 作者
\institute{LeNotFound OI Team}  % 组织
\date{2023-02-20}   % 日期
\AtBeginSection[]{
  \begin{frame}
    \frametitle{目录}
    \tableofcontents[currentsection]
  \end{frame}
}

\begin{document}

\frame{\titlepage} % 生成标题页

\section{第一章} % 一级章节标题

\subsection{第一节} % 二级章节标题

\begin{frame} % 插入空白页

    \frametitle{页面标题} % 一页的标题

    给定一个 01 串,问能否在 $K$ 次操作内统一为 $0$ 或 $1$。\\
    ~\
    \pause

    为什么说到 01 串呢,原题不是 $top$ 和 $bottom$ 吗?

\end{frame}

\begin{frame}
    \frametitle{页面标题}

    这就引入一个新的概念:\\
    ~\
    \begin{center}  % 居中
        \textbf{状态压缩}   % 粗体
    \end{center}

    ~\  % 空行

    \pause  % 暂停

    实际上状态压缩常用于状压 DP,用二进制上的每一个数对应集合的一种状态,节省空间,同时可以通过位运算提高效率。

\end{frame}

\subsection{第二节} % 二级章节标题

\begin{frame}[fragile] % 包含代码块的页面
    \frametitle{页面标题}

    以流的方式输入输出,本题主要使用 \verb|istringstream|。\\
    
    ~\

    \verb|xxxxx| % 行内代码,等同于 markdown `xxxxx`

\end{frame}

\begin{frame}[fragile]
    \frametitle{页面标题}
    % 代码块,可标记语言高亮
    \begin{lstlisting}[language=C++]
        string str;
        while(cin>>str)
        {
            // do something...
        }
    \end{lstlisting}

\end{frame}

\section{第二章}

\subsection{第一节}

\begin{frame}
    \frametitle{页面标题}

    \includegraphics[height=4cm]{pic.jpg} % 插入图片

    % 图片可调整宽高

\end{frame}

\subsection{第二节}

\begin{frame}
    \frametitle{页面标题}
\end{frame}

\end{document}